\documentclass{article}
\usepackage[spanish]{babel}
\usepackage{amsmath}
\usepackage{amssymb}
\usepackage{geometry}
\geometry{a4paper, margin=2cm}
\usepackage{booktabs}

\title{Aplicación de la Ley de Amdahl a Configuraciones Multi-GPU}
\author{Análisis de Rig Minero}
\date{}

\begin{document}

\maketitle

\section{Ley de Amdahl}
\subsection*{Fórmula Adaptada a GPUs}

\subsection*{Formulación General}
La aceleración (\(S\)) en sistemas multi-GPU se calcula como:
\[
\boxed{S = \dfrac{1}{(1 - P) + \dfrac{P}{N}}}
\]
donde:
\begin{itemize}
  \item \(S\): \textbf{Aceleración real} (vs. rendimiento con 1 GPU)
  \item \(P\): Fracción \textbf{paralelizable} del algoritmo (\(0 \leq P \leq 1\))
  \item \(N\): Número de \textbf{GPUs} (\(N = 3\) en su configuración)
\end{itemize}

\subsection*{Interpretación Física}
\begin{align*}
\text{Límite teórico:} \quad & \lim_{N \to \infty} S = \frac{1}{1 - P} \\
\text{Eficiencia:} \quad & E = \frac{S}{N} \times 100\%
\end{align*}

\subsection*{Aplicación Práctica al Rig}

\subsection*{Escenario Ideal (\(P = 0.99\))}
\begin{itemize}
  \item Paralelización declarada: \(P = 0.99\)
  \item Número de GPUs: \(N = 3\)
\end{itemize}
\[
S = \dfrac{1}{(1 - 0.99) + \dfrac{0.99}{3}} = \dfrac{1}{0.01 + 0.33} = \dfrac{1}{0.34} \approx 2.94\times
\]

\begin{table}[h]
\centering
\caption{Rendimiento teórico vs. real}
\begin{tabular}{lccc}
\toprule
Métrica & 1 GPU & 3 GPUs (Teórico) & 3 GPUs (Real) \\
\midrule
Hashrate (MH/s) & 60 & 180 & 176.4 \\
Aceleración (\(S\)) & \(1\times\) & \(3\times\) & \(2.94\times\) \\
Eficiencia (\(E\)) & 100\% & 100\% & 98\% \\
\bottomrule
\end{tabular}
\end{table}

\subsection*{Escenario Realista (\(P_{\text{real}} = 0.95\))}
\begin{itemize}
  \item \textbf{Factores reductores:}
  \begin{itemize}
    \item Sincronización entre GPUs (\(\downarrow 1.5\%\))
    \item Overhead de drivers (\(\downarrow 2\%\))
    \item Límites PCIe (\(\downarrow 1\%\))
  \end{itemize}
  \item Paralelización efectiva: \(P_{\text{real}} = 0.95\)
\end{itemize}
\[
S = \dfrac{1}{(1 - 0.95) + \dfrac{0.95}{3}} = \dfrac{1}{0.05 + 0.316} = \dfrac{1}{0.366} \approx 2.73\times
\]

\vspace{1em}
\noindent
\textbf{Impacto en rendimiento:}
\begin{align*}
\text{Rendimiento esperado:} \quad & 180\ \text{MH/s} \\
\text{Rendimiento real:} \quad & 60 \times 2.73 = 163.8\ \text{MH/s} \\
\text{Pérdida:} \quad & \Delta = \left(1 - \frac{163.8}{180}\right) \times 100\% \approx 9\%
\end{align*}

\end{document}